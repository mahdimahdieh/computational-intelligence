% Exercise Template
%A LaTeX template for typesetting exercise in Persian (with cover page).
%By: Zeinab Seifoori

\documentclass[12pt]{exam}

\usepackage{setspace}
\usepackage{listings}
\usepackage{xcolor}

% Define colors
\definecolor{codegreen}{rgb}{0,0.6,0}
\definecolor{codegray}{rgb}{0.5,0.5,0.5}
\definecolor{codepurple}{rgb}{0.58,0,0.82}
\definecolor{backcolour}{rgb}{0.95,0.95,0.92}


% Configure listings style
\lstdefinestyle{mystyle}{
	backgroundcolor=\color{backcolour},   
	commentstyle=\color{codegreen},
	keywordstyle=\color{magenta},
	numberstyle=\tiny\color{codegray},
	stringstyle=\color{codepurple},
	basicstyle=\ttfamily\footnotesize\setLTR,
	breakatwhitespace=false,         
	breaklines=true,                 
	captionpos=top, % Changed to top placement
	keepspaces=true,                 
	numbers=left,                    
	numbersep=5pt,                  
	showspaces=false,                
	showstringspaces=false,
	showtabs=false,                  
	tabsize=2,
	frame=single,
	abovecaptionskip=10pt, % Space between caption and code
	belowcaptionskip=10pt, % Space between code and potential bottom text
}


\lstset{style=mystyle}

% Set Persian caption for listings
\renewcommand{\lstlistingname}{برنامه}

\usepackage{graphicx,subfigure,wrapfig}
\usepackage{multirow}
%\usepackage{multicol}

\usepackage[margin=20mm]{geometry}
\usepackage{hyperref}
\hypersetup{
	colorlinks=true,
	linkcolor=blue,
	filecolor=magenta,      
	urlcolor=cyan,
	pdftitle={Overleaf Example},
	pdfpagemode=FullScreen,
}
\usepackage{xepersian}
\settextfont{XB Niloofar}

\newcommand{\class}{درس مبانی هوش محاسباتی}
\newcommand{\term}{نیم‌سال دوم ۰۱-۰۲}
\newcommand{\college}{دانشکده مهندسی کامپیوتر}
\newcommand{\prof}{استاد: دکتر کارشناس}

\singlespacing
\parindent 0ex

\begin{document}


% -------------------------------------------------------
%  Thesis Information
% -------------------------------------------------------

\newcommand{\ThesisType}
{سمینار}  % پایان‌نامه / رساله
\newcommand{\ThesisDegree}
{کارشناسی}  % کارشناسی / کارشناسی ارشد / دکتری
\newcommand{\ThesisMajor}
{مهندسی کامپیوتر}  % مهندسی کامپیوتر
\newcommand{\ThesisTitle}
{فاز دوم تمرین سوم هوش محاسباتی: 
	
	طراحی سیستم منطق فازی برای توصیه تمرینات ورزشی}
\newcommand{\ThesisAuthor}
{دانیال شفیعی
	
مهدی مهدیه

امیررضا نجفی}
\newcommand{\ThesisSupervisor}
{دکتر کارشناس}
\newcommand{\ThesisDate}
{اردیبهشت 1404}
\newcommand{\ThesisDepartment}
{دانشکده مهندسی کامپیوتر}
\newcommand{\ThesisUniversity}
{دانشگاه اصفهان}



\pagestyle{empty}
\include{cover-page}

% These commands set up the running header on the top of the exam pages
\pagestyle{head}
\firstpageheader{}{}{}
\runningheader{صفحه \thepage\ از \numpages}{}{\class}
\runningheadrule
%\begin{tabular}{p{.7\textwidth} l}
%\multicolumn{2}{c}{\textbf{به نام خدا}}\\
%\multirow{2}{*}{\includegraphics[scale=0.2] {images/logo.png}} & \\ \\
%&  \textbf{\class}\\
%&  \textbf{\term}\\
%&  \textbf{\prof}\\ \\
% \textbf{\college} &  \\
%\end{tabular}\\

%\rule[1ex]{\textwidth}{.1pt}
%\textbf{تمرین سری پنجم}

%\rule[1ex]{\textwidth}{.1pt}
%\makebox[45mm]{\hrulefill}\\
\tableofcontents
\vspace{1cm}
\setcounter{section}{-1}
\section{مقدمه}
هدف از این تمرین آشنایی بیشتر با الگوریتم‌های ژنتیک و استفاده‌ی بیشتر از آن‌ها در کاربردهای عملی است.
\section{
	مبانی و مفاهیم الگوریتم ژنتیک}
	
	
\section{
	درک و حل مسائل با الگوریتم ژنتیک}
	\begin{questions}
		\question
		\begin{parts}
			\part طول
			$n\times\frac{n - 1}{2} $
			\part
		\end{parts}
		
		\question
		\begin{parts}
			\part ژن‌ها  را به تابع 
			fitness
			 می‌بریم:
			\[\mathrm{fit} (x_1) = 6+5-4-1+3+5-3-2=9 \]
			\[\mathrm{fit} (x_2) = 8+7-1-2+6+6-0-1=23\]
			\[\mathrm{fit} (x_3) = 2+3-9-2+1+2-8-5=-16\]
			\[\mathrm{fit} (x_4) = 4+1-8-5+2+0-9-4=-19 \]
			به ترتیب
			$x_2$، $x_1$،$x_3$ و $x_4$
			 برازنده هستند.
			 
			 \part عملیات ترکیب
			 \begin{itemize}
			 	\item
			 	 ترکیب نقطه‌ای: در این روش به دو فرزند جدید می‌رسیم.
			 	 \[x_{21} = 8712|3532\]
			 	 \[x_{21}=6541|6601\]
			 	 \item
			 	  ترکیب دو نقطه‌ای: با استفاده از این روش به دو فرزند جدید می‌رسیم. ما فرض می‌کنیم منظور از نقاط b و f یعنی بعد از این نقاط ترکیب اتفاق می‌افتد
			 	  \[x_{131} = 65|9212|35\]
			 	  \[x_{313}=23|4135|85\]
			 	  \item 
			 	  ترکیب یکنواخت: برای انجام این ترکیب نیازمند به یک ماسک هستیم. این ماسک یک ژن تصادفی با مقادیر دودویی است که نشانگر این است که آن ژن را از کروموزوم اول بگیریم یا دوم. که انتخاب اول یا دوم هم احتمال است. ما با استفاده از
			 	  

\begin{lstlisting}[language=Python, caption= تولید ماسک تصادفی]
	mask = ''.join(random.choice('01') for _ in range(8))
\end{lstlisting}

			 	  	
			 	  
			 	  یک رشته‌ی تصادفی از ۰ و ۱ تولید می‌کنیم. ما فرض می‌کنیم ۰  معادل رشته‌ی اول و ۱ معادل رشته‌ی سوم باشد.
			 	  
			 	  \[\mathrm{mask} = 01001010\]
			 	  \[x_{13} = 8|3|12|1|6|8|1\]
			 	  \[x_{31}=2|7|92|6|2|0|5\]
			 \end{itemize}
			 \part
			  برازش فرزندان: با استفاده از تکه کد زیر برازندگی هر فرزند را محاسبه می‌کنیم:
			  
			 
			 \begin{lstlisting}[language=Python, caption= محاسبه‌ی برازندگی]
			 	chromosome = input()
			 	a, b, c, d, e, f, g, h = [int(char) for char in chromosome]
			 	fitness = a + b - c - d + e + f - g - h
			 	print(fitness)
			 \end{lstlisting}
			  \[\mathrm{fit} (x_{21} )= 87123532=15\]
			 \[\mathrm{fit} (x_{21})= 65416601=17\]
			 \[\mathrm{fit} (x_{131}) = 65921235 = -5\]
			 \[\mathrm{fit} (x_{313}) = 23413585 = -5\]
			   \[\mathrm{fit} (x_{23}) = 83121681 = 6\]
			 \[\mathrm{fit} (x_{32}) = 27926205 = 1\]
		\end{parts} 
		
	\end{questions}
	
\section[
پیاده‌سازی، ارزیابی و تجزیه‌ و تحلیل الگوریتم ژنتیک]{پیاده‌سازی، ارزیابی و تجزیه‌ و تحلیل الگوریتم ژنتیک جهت انتخاب بهترین ویژگی برای مسئله‌ی واقعی دسته‌بندی مشتریان} 
\end{document}