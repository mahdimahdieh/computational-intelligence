% Exercise Template
%A LaTeX template for typesetting exercise in Persian (with cover page).
%By: Zeinab Seifoori

\documentclass[12pt]{exam}

\usepackage{setspace}
\usepackage{listings}
\usepackage{xcolor}

% Define colors
\definecolor{codegreen}{rgb}{0,0.6,0}
\definecolor{codegray}{rgb}{0.5,0.5,0.5}
\definecolor{codepurple}{rgb}{0.58,0,0.82}
\definecolor{backcolour}{rgb}{0.95,0.95,0.92}


% Configure listings style
\lstdefinestyle{mystyle}{
	backgroundcolor=\color{backcolour},   
	commentstyle=\color{codegreen},
	keywordstyle=\color{magenta},
	numberstyle=\tiny\color{codegray},
	stringstyle=\color{codepurple},
	basicstyle=\ttfamily\footnotesize\setLTR,
	breakatwhitespace=true,         
	breaklines=true,                 
	captionpos=top, % Changed to top placement
	keepspaces=true,                 
	numbers=left,                    
	numbersep=5pt,                  
	showspaces=false,                
	showstringspaces=false,
	showtabs=false,                  
	tabsize=2,
	frame=single,
	abovecaptionskip=5pt, % Space between caption and code
	belowcaptionskip=5pt, % Space between code and potential bottom text
}


\lstset{style=mystyle}

% Set Persian caption for listings
\renewcommand{\lstlistingname}{برنامه}

\usepackage{graphicx,subfigure,wrapfig}
\usepackage{float} %for floating shapes

\usepackage{multirow}
%\usepackage{multicol}


%%%%%%%%%  for pie chart %%%%%
\usepackage{pgf-pie}  
%%%%%%%%%%%%%%%%%%%%%%%%%%%%

%%%%%%%% for qoute block %%%%%
\usepackage{etoolbox}
%%%%%%%%%%%%%%%%%%%%%%5


\usepackage[margin=20mm]{geometry}
\usepackage{hyperref}
\hypersetup{
	colorlinks=true,
	linkcolor=blue,
	filecolor=magenta,      
	urlcolor=cyan,
	pdftitle={Overleaf Example},
	pdfpagemode=FullScreen,
}
\usepackage{xepersian}
\settextfont{XB Niloofar}

\newcommand{\class}{درس مبانی هوش محاسباتی}
\newcommand{\term}{نیم‌سال دوم ۰۱-۰۲}
\newcommand{\college}{دانشکده مهندسی کامپیوتر}
\newcommand{\prof}{استاد: دکتر کارشناس}

\onehalfspacing
\parindent 0ex
\begin{document}
	
	
% -------------------------------------------------------
%  Thesis Information
% -------------------------------------------------------

\newcommand{\ThesisType}
{سمینار}  % پایان‌نامه / رساله
\newcommand{\ThesisDegree}
{کارشناسی}  % کارشناسی / کارشناسی ارشد / دکتری
\newcommand{\ThesisMajor}
{مهندسی کامپیوتر}  % مهندسی کامپیوتر
\newcommand{\ThesisTitle}
{فاز دوم تمرین سوم هوش محاسباتی: 
	
	طراحی سیستم منطق فازی برای توصیه تمرینات ورزشی}
\newcommand{\ThesisAuthor}
{دانیال شفیعی
	
مهدی مهدیه

امیررضا نجفی}
\newcommand{\ThesisSupervisor}
{دکتر کارشناس}
\newcommand{\ThesisDate}
{اردیبهشت 1404}
\newcommand{\ThesisDepartment}
{دانشکده مهندسی کامپیوتر}
\newcommand{\ThesisUniversity}
{دانشگاه اصفهان}



	\pagestyle{empty}
	\include{cover-page}
	
	% These commands set up the running header on the top of the exam pages
	\pagestyle{head}
	\firstpageheader{}{}{}
	\runningheader{صفحه \thepage\ از \numpages}{}{\class}
	\runningheadrule
	%\begin{tabular}{p{.7\textwidth} l}
	%\multicolumn{2}{c}{\textbf{به نام خدا}}\\
	%\multirow{2}{*}{\includegraphics[scale=0.2] {images/logo.png}} & \\ \\
	%&  \textbf{\class}\\
	%&  \textbf{\term}\\
	%&  \textbf{\prof}\\ \\
	% \textbf{\college} &  \\
	%\end{tabular}\\
	
	%\rule[1ex]{\textwidth}{.1pt}
	%\textbf{تمرین سری پنجم}
	
	%\rule[1ex]{\textwidth}{.1pt}
	%\makebox[45mm]{\hrulefill}\\
	\tableofcontents
	\vspace{1cm}
	\setcounter{section}{-1}
	\newpage
	\section{چکیده}
	در این پروژه، یک سیستم منطق فازی برای توصیه شدت و مدت زمان تمرینات ورزشی بر اساس ویژگی‌های فردی طراحی و پیاده‌سازی شد. ورودی‌های سیستم شامل سطح آمادگی جسمانی، سطح انرژی، هدف ورزشی، وزن و سن بود و خروجی‌ها شدت تمرین و مدت زمان آن بودند. با استفاده از \lr{scikit-fuzzy} در زبان \lr{Python}، توابع عضویت و قواعد فازی تعریف شده و نتایج سیستم با چند مثال مختلف آزمایش و تحلیل شد.
	
	\section{مقدمه}
	امروزه استفاده از سیستم‌های هوشمند در حوزه سلامت و ورزش گسترش یافته است. سیستم فازی به دلیل توانایی مدل‌سازی رفتارهای «نرم» و «نامعین» مناسب‌‌ترین گزینه برای توصیه برنامه تمرینی شخصی‌شده است. در این گزارش، اهداف، روش پیاده‌سازی، تعریف متغیرها و قواعد، و نتایج تست‌ها بیان می‌شود.
	
	\section{روش پیاده‌سازی}
	\begin{itemize}
		
		\item کتابخانه‌های مورد استفاده: \lr{numpy}, \lr{scikit-fuzzy}, \lr{matplotlib}
		\item ساختار کد:
		\begin{enumerate}
			\item تابع \texttt{define\_variables} برای تعریف متغیرهای فازی و توابع عضویت
			\item تابع \texttt{define\_rules} برای تعریف قواعد فازی
			\item تابع \texttt{build\_and\_run\_system} برای اجرای شبیه‌سازی و اعمال شخصی‌سازی
			\item تابع \texttt{plot\_memberships\_and\_results} برای رسم توابع عضویت و نتایج
			\item تابع \texttt{main} برای مدیریت ورودی کاربر، اجرای تست‌ها و چاپ نتایج
		\end{enumerate}
		\item قابلیت‌های اضافی:
		\begin{itemize}
			\item کاهش خودکار شدت تمرین برای افراد با سن بیش تر از 60 سال
			\item پرسش از کاربر جهت ترجیح شدت بیشتر (ضریب \lr{1.2} بر شدت نهایی)
			\item نشان دادن نمودار های توابع  عضویت در حالت کلی و پس از محاسبات
		\end{itemize}
	\end{itemize}
	
	\section{تعریف متغیرهای فازی}
	\subsection{ورودی‌ها}
	\begin{itemize}
		\item \textbf{physical\_fitness} (\lr{0 – 100}): \lr{beginner, intermediate, advanced }
		\item \textbf{energy} (\lr{0 – 100}): \lr{low, medium, high}  
		\item \textbf{sports\_goal} (\lr{0 - 2}): \lr{weight\_loss, muscle\_gain, general\_fitness}
		\item \textbf{weight} (\lr{40 - 120 kg}): \lr{underweight, normal, overweight}  
		\item \textbf{age} (\lr{15 - 80} سال): \lr{young, middle, elderly} (برای پردازش پس از استنتاج)
	\end{itemize}
	
	\subsection{خروجی‌ها}
	\begin{itemize}
		\item \textbf{exercise\_intensity} (\lr{0 - 100}): \lr{low, medium, high}  
		\item \textbf{exercise\_time} (\lr{0 – 120} دقیقه): \lr{short, medium, long}  
	\end{itemize}
	
	توابع عضویت هر متغیر به صورت مثلثی (\lr{trimf}) و با بازه‌های هم‌پوشان تعریف شده تا نرمی تصمیم‌گیری حفظ شود.
	
	\section{قواعد فازی}
	در مجموع ۷ قانون فازی به شرح زیر تعریف شده‌اند:
	\begin{enumerate}
		\item اگر سطح آمادگی جسمانی \textbf{beginner} و سطح انرژی \textbf{low} باشد،  
		آنگاه شدت \textbf{low} و مدت \textbf{short}.
		\item اگر سطح آمادگی جسمانی \textbf{beginner} و سطح انرژی \textbf{medium} باشد،  
		آنگاه شدت \textbf{low} و مدت \textbf{medium}.
		\item اگر سطح آمادگی جسمانی \textbf{intermediate} و سطح انرژی \textbf{high} باشد،  
		آنگاه شدت \textbf{medium} و مدت \textbf{medium}.
		\item اگر سطح آمادگی جسمانی \textbf{advanced} و سطح انرژی \textbf{high} باشد،  
		آنگاه شدت \textbf{high} و مدت \textbf{long}.
		\item اگر هدف ورزشی \textbf{weight\_loss} و سطح انرژی \textbf{high} باشد،  
		آنگاه شدت \textbf{medium} و مدت \textbf{long}.
		\item اگر وزن \textbf{overweight} و هدف ورزشی \textbf{weight\_loss} باشد،  
		آنگاه مدت \textbf{long}.
		\item اگر هدف ورزشی \textbf{muscle\_gain} و سطح آمادگی جسمانی \textbf{advanced} باشد،  
		آنگاه شدت \textbf{high}.
	\end{enumerate}
	
	\section{نتایج تست‌ها}
	برای مثال های زیر اجرا و محاسبات انجام شد و خروجی‌ها به‌صورت میانگین نهایی گزارش گردید:
	
	\begin{itemize}
		\item ورودی:
		\lr{\{physical\_fitness=20, energy=30, sports\_goal=0, age=25, weight=65\}}\\
		خروجی:
		\lr{intensity=16.2، time=22.5 minutes}
		\item ورودی:
		\lr{\{physical\_fitness=70, energy=80, sports\_goal=1, age=30, weight=75\}}\\
		خروجی:
		\lr{intensity=92.1، time=83.7 minutes}
		\item ورودی:
		\lr{\{physical\_fitness=40, energy=50, sports\_goal=0, age=65, weight=80\}}\\
		خروجی:
		\lr{intensity=20.3، time=57.9 minutes}
		\item ورودی:
		\lr{\{physical\_fitness=90, energy=90, sports\_goal=2, age=40, weight=60\}}\\
		خروجی:
		\lr{intensity=88.4، time=102.3 minutes}
		\item ورودی: 
		\lr{\{physical\_fitness=30, energy=20, sports\_goal=0, age=70, weight=90\}}\\
		خروجی:
		\lr{intensity=12.6، time=35.2}
	\end{itemize}
	
	خروجی‌ها باشرایط هر فرد و نیازهای تمرینی وی همخوانی مناسبی دارند. به‌خصوص کاهش شدت برای افراد مسن و در صورت تمایل کاربر به تمرین شدیدتر، شدت نهایی افزایش می‌یابد این عمل از کاربر در ترمنینال پرسیده می شود.
	
	\section{نتیجه‌گیری}
	این سیستم فازی توانست برای ترکیب‌های مختلف ویژگی‌های فردی، شدت و زمان تمرین منطقی ارائه دهد. منطق فازی با توابع عضویت مثلثی و قواعد ساده توانایی مدل‌سازی نرمی تصمیم‌گیری را دارد. توابع عضویت را می توان با استفاده از داده های آماری و شرایط هر منطقه خصوصی سازی و بر اساس داده‌های واقعی تنظیم کرد.

\end{document}