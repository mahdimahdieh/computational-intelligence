


		\subsection{ پیش پردازش داده‌ها}
	
		\begin{parts}
			\part
			\textbf{
					حذف داده‌های پرت:
			}
		
					
			برای پر کردن داده‌های پرت از روش 
		\lr{IQR method}
			استفاده می‌کنیم این روش به این صورت است که 
			\lr{IQR}
			را برابر با 
			\lr{Q3 - Q1}
			قرار می‌دهیم(چارک اول : 
			\lr{Q1},
			چارک سوم:
			\lr{Q3}
			)
			 سپس داده‌های کوچک‌تر از 
			$\mathrm{Q1 }- 1.5\times \mathrm{IQR}$
			و بزرگ‌تر از 
		 $\mathrm{Q3} + 1.5\times \mathrm{IQR}$
			را حذف می‌کنیم.
			\lstinputlisting[language=Python, caption= حذف داده‌ها‌ی پرت]{./scripts/IQR.py}		
			\part
			\textbf{
			رمزگذاری ویژگی‌های دسته‌ای
			\lr{(categorical)}: }
			
			مهم‌ترین پارامتر در رمزگذاری
			\lr{(encoding)}
			 داده‌های دسته‌ای این است که ببینیم که این داده‌ها داده‌های کیفی ترتیبی 
			 \lr{(Ordinal Qualitative)}
			 هستند یا کیفی اسمی
			 \lr{(Nominal Qualitative)}
			 اگر داده‌ها کیفی ترتیبی باشند نیاز داریم که داده‌ها را به گونه‌ای پیش‌پردازش کنیم که این ترتیب همچنان حفظ شود و در صورتی که داده‌ها اسمی باشند نیازی به این کار نداریم و می‌توانیم برای هر دسته یک ستون درست کنیم که با مقادیر درست 
			 \lr{(True)}
	و غلط
	  \lr{(False)}
	  مشخص کنیم که به این دسته قرار دارد یا خیر.
	  
	  در داده‌هایی که در سوال به ما داده شده بود ستون‌های 
	  \lr{Gender}، 
	  \lr{Ever\_Married}
	  ، 
	  \lr{Graduated}
	  ، 
	  \lr{Profession}
	  و 
	  \lr{Var\_1}
	  ستون‌هایی بودند که داده‌های آنها به صورت کیفی اسمی بود و برای همین برای رمز گذاری آنها از 
	  \lr{get\_dummies}
	   استفاده کردیم.

	   
	   \lstinputlisting[language=Python, caption=رمزگذاری با  
	   \lr{get\_dummies} ]{./scripts/get_dummies.py}
	   
	   
	   و ستون 
	   \lr{Spending\_Score}
	    را که دارای داده‌های ترتیبی به نام‌های 
	   \lr{Low}،
	    \lr{Average} و 
	     \lr{High}
	     بود را به صورت دیگری رمزگذاری کردیم تا مدل توانایی درک این که این 3 مقدار دارای ترتیب مشخصی هستند را متوجه شود.    اینکار را به این صورت انجام دادیم که این 3 ستون را به ترتیب از کوچک به بزرگ از 0 تا 2 مقداردهی کردیم.
	     
	     \lstinputlisting[language=Python, caption=رمزگذاری برای ستون 
	     \lr{spending\_score}]{./scripts/spending_score.py}
	     
	\part
	\textbf{
			پر کردن داده‌های خالی 
				\lr{(Nan)}:
	}


	
	برای پرکردن داده‌های خالی از 
	\lr{KNNImputer}
	در کتابخانه sikitlearn استفاده کردیم به این صورت که 
	\lr{n\_neighbors} که یکی از پارامترهای این تابع است را 8 انتخاب کردیم و این به این معناست که هر ردیفی که مفدار خالی داشته باشد میاید و 8 ردیف نزدیک به ان را پیدا می‌کند و سپس از مقادیر ستون مورد نظر(یعنی ستونی که مقدار خالی در ان قرار دارد.)
	در آن ردیف‌ها میانگین گرفته و آنرا به عنوان مقدار جدید سلول خالی قرار می‌دهد.
	\lstinputlisting[language=Python, caption= پر کردن سلول‌های خالی]{./scripts/filling_nans.py}
	
	\end{parts}
	

\subsection{پیاده‌سازی الگوریتم ژنتیک}
	
	الگوریتم ژنتیک یک روش جستجو و بهینه‌سازی مبتنی بر تکامل طبیعی است که در آن مفاهیمی از ژنتیک مانند انتخاب طبیعی، ترکیب و جهش برای یافتن بهترین جواب به کار گرفته می‌شوند. این الگوریتم در مسائل مختلف از جمله بهینه‌سازی، یادگیری ماشین و مسائل ترکیبیاتی مورد استفاده قرار می‌گیرد.
	
	
	\subsubsection{مراحل الگوریتم ژنتیک}

	\begin{parts}
	\part
	\textbf{
			مقداردهی اولیه 
		\lr{(Initialization)}:
	}

	
	مجموعه‌ای از کروموزوم‌ها (یا راه‌حل‌های ممکن) به‌صورت تصادفی ایجاد می‌شود.
	 \lstinputlisting[language=Python, caption=مقداردهی اولیه]{./scripts/initialize.py}
	 
	 \part
	 \textbf{
	 ارزیابی برازندگی
	 \lr{(Fitness Evaluation)}:
	}
	 
	 هر کروموزوم بر اساس یک تابع برازندگی ارزیابی می‌شود تا میزان تطابق آن با هدف مشخص شود. در کدی که ما زدیم تابع برازندگی، 
	 \lr{accuracy\_score}
	  مدل است.
	 \part
	 \textbf{
	 انتخاب
	 \lr{(Selection)}:
	}
	
	 کروموزوم‌های بهتر شانس بیشتری برای انتخاب شدن و انتقال به نسل بعدی دارند. روش‌های مختلفی برای این کار وجود دارد، از جمله:
	 
	 \begin{itemize}
	 	\item 	 چرخ رولت 
	 	\lr{(Roulette Wheel Selection)}
	 	
	 	\item 
	 	انتخاب بر اساس رتبه
	 	\lr{(Rank-Based Selection)}
	 	
	 	\item  انتخاب تورنمنت
	 	\lr{(Tournament Selection)}
	 	
	 	
	 \end{itemize}

	 
	  
	  
	
	  که ما هر ۳تای انها را پیاده‌سازی کردیم.
	  
	  
	  \lstinputlisting[language=Python, caption=roulette-wheel-selection]{./scripts/roulette_wheel_selection.py}
	  
	  \lstinputlisting[language=Python, caption=rank-based-selection]{./scripts/rank_based_selection.py}
	  
	  \lstinputlisting[language=Python, caption=tournament-selection]{./scripts/tournament_selection.py}
	  
	  	\begin{figure}[H]
	  	\centering
	  	\includegraphics[width=1\textwidth]{images/decision-tree}
	  	\caption{درخت تصمیم }
	  \end{figure}
	  
	  \part
	  \textbf {
	  	  ترکیب
	  \lr{(Crossover)}:
	  }

	  
	  
	  کروموزوم‌های انتخاب شده با یکدیگر ترکیب می‌شوند تا فرزندان جدید تولید شود. روش‌های متداول شامل:
	  
	  \begin{itemize}
	  	\item
	  ترکیب تک‌نقطه‌ای
	  \lr{(Single-Point Crossover)}
	  \item
	  ترکیب دو‌نقطه‌ای 
	  \lr{(Two-Point Crossover)}
	  \item
	  ترکیب یکنواخت
	  \lr{(Uniform Crossover)}
	  \end{itemize}
	  که ما هر ۳تای این‌ها را هم پیاده‌سازی کردیم.

	  
	  \lstinputlisting[language=Python, caption=single-point-crossover]{./scripts/single_point_crossover.py}
	  
	  \lstinputlisting[language=Python, caption=two-point-crossover]{./scripts/two_point_crossover.py}
	  
	  \lstinputlisting[language=Python, caption=uniform-crossover]{./scripts/uniform_crossover.py}
	  
	  \part
	  \textbf{
	  جهش
	  	\lr{(Mutation)}:
		  }
	   
	  برخی از کروموزوم‌ها دچار تغییرات جزئی می‌شوند تا از هم‌گرایی زودرس جلوگیری شود و تنوع حفظ گردد.
	 
	 \lstinputlisting[language=Python, caption=mutate]{./scripts/mutate.py}
	  
	\part 
	\textbf{
			تکرار
		\lr{(Iteration)}:
	}

	
	همه‌ی مراحل الف تا ه را دوباره انجام می‌دهیم تا به یکی از دو شرط زیر برسیم:
	\begin{enumerate}
		\item رسیدن به حداکثر تعداد تکرار
		\lr{(iterations)}
		\item رسیدن به نقطه‌ی همگرایی‌ (عدم بهبود برای مدت طولانی)
	\end{enumerate}
	
	
	
	حالا با استفاده از الگوریتم‌هایی که در بالا توضیح داده و پیاده سازی کردیم، الگوریتم ژنتیک را پیاده‌سازی می‌کنیم.
	
	
	\lstinputlisting[language=Python, caption=Genetic Algorithm]{./scripts/GA.py}
	
	
\part
	خروجی الگوریتم
	\begin{figure}[H]
		\centering
		\includegraphics[width=0.8\textwidth]{images/steps}
		\caption{روند بهبود عملکرد الگوریتم در طی نسل‌ها}
	\end{figure}
	\end{parts}
