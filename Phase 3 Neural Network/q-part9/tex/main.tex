% !TEX program = xelatex
\documentclass[12pt]{exam}
\usepackage{geometry}
\usepackage{xepersian}
\settextfont{XB Niloofar}
\usepackage{amsmath, amssymb}
\usepackage{graphicx}
\usepackage{hyperref}
\begin{document}
	\begin{center}
		\LARGE جواب سوال 9
	\end{center}
	
	\noindent
	\textbf{صورت پرسش:}
	
	یک شبکه‌عصبی دو لایه را در نظر بگیرید که ورودی‌های آن از فضای دوبعدی $\mathbb{R}^2$ هستند. هدف این است که این فضا به $m$ ناحیهٔ مجزا تقسیم شود. حداقل چند نورون در لایهٔ پنهان لازم است تا این تقسیم‌بندی ممکن شود؟
	
	\vspace{1em}
	\noindent
	\textbf{پاسخ:}
	
	برای تقسيم $\mathbb{R}^2$ به $m$ ناحيهٔ مجزا با يک شبکهٔ دو لايه، بايد $k$ نورون در لايهٔ پنهان داشته باشيم به طوري که حداکثر تعداد ناحيه‌هاي ايجادشده توسط $k$ خط برابر باشد با:
	\[
	N(k)=\frac{k(k+1)}{2}+1.
	\]
	پس بايد عدد $k$ کمترين عدد صحيح باشد که
	\[
	N(k)\ge m\quad\Longrightarrow\quad \frac{k(k+1)}{2}+1\ge m.
	\]
	با حل اين نامعادله، خواهيم داشت:
	\[
	k^2 + k -2(m-1)\ge 0
	\quad\Longrightarrow\quad k\ge \frac{-1+\sqrt{1+8(m-1)}}{2}.
	\]
	بنابراين حداقل تعداد نورون‌هاي لايهٔ پنهان:
	\[
	k=\left\lceil\frac{-1+\sqrt{8m-7}}{2}\right\rceil.
	\]
	
	\vspace{1em}
	\noindent
	\textbf{اثبات فرمول حداکثر تعداد ناحیه‌ها با استفاده از استقرا:}
	
	\begin{description}
		\item[قضيه:] با $k$ خط در صفحه حداکثر $N(k)=k(k+1)/2+1$ ناحيه متناظر ايجاد مي‌شود.
		\item[پايهٔ استقرا ($k=0$):] با صفر خط در صفحه، تمام صفحه يک ناحيهٔ يکپارچه است. بنابراين
		\[
		N(0)=\frac{0\cdot1}{2}+1=1,
		\]
		که صحيح است.
		\item[فرض استقرا:] فرض کنيم براي $k$ خط، فرمول
		\[
		N(k)=\frac{k(k+1)}{2}+1
		\]
		برقرار باشد.
		\item[گام استقرا ($k\to k+1$):]
		اگر يک خط تازه به مجموعهٔ $k$ خط اضافه کنيم، اين خط جديد با هر يک از $k$ خط قبلي در يک نقطه تلاقی مي‌کند، بنابراين در مجموع $k$ نقطهٔ تلاقی ايجاد مي‌شود. اين $k$ نقطه، خط جديد را به $k+1$ قطعه تقسيم مي‌کند. هر قطعه يک ناحيهٔ جديد ايجاد مي‌کند. بنابراين:
		\[
		N(k+1)=N(k)+(k+1).
		\]
		با جاگذاري فرض استقرايي داريم:
		\[
		N(k+1)=\frac{k(k+1)}{2}+1+(k+1)
		=\frac{k(k+1)+2(k+1)}{2}+1
		=\frac{(k+1)(k+2)}{2}+1.
		\]
		بنابراين فرمول براي $k+1$ نيز برقرار است.
	\end{description}
	
	\noindent
	اين اثبات کامل است و نتيجه مي‌دهد که براي تقسيم $\mathbb{R}^2$ به $m$ ناحيه، حداقل
	\[
	\left\lceil\tfrac{-1+\sqrt{8m-7}}{2}\right\rceil
	\]
	نورون در لايهٔ پنهان نياز داريم.
	
\end{document}
