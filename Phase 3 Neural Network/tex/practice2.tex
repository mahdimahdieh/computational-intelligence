\section{ کدزنی و پیاده سازی‬}
\begin{enumerate}
	\item \textbf{گزارش پیاده‌سازی رگرسیون لجستیک برای طبقه‌بندی دودویی تصاویر CIFAR-10}
	
	\textbf{مقدمه}
	
	این گزارش به جزئیات پیاده‌سازی و ارزیابی یک مدل رگرسیون لجستیک از پایه در پایتون می‌پردازد. هدف اصلی، طبقه‌بندی دودویی تصاویر از مجموعه داده CIFAR-10 است، به این صورت که تشخیص دهد آیا تصویر "هواپیما" (کلاس 0) است یا "سایر کلاس‌ها". این پیاده‌سازی شامل مراحل بارگذاری داده، پیش‌پردازش، تعریف توابع فعال‌سازی و زیان، پیاده‌سازی مدل رگرسیون لجستیک با به‌روزرسانی گرادیان کاهشی و در نهایت ارزیابی عملکرد مدل است.
	
	\textbf{1. بارگذاری و نرمال‌سازی دیتاست CIFAR-10}
	
	مجموعه داده CIFAR-10 با استفاده از \texttt{torchvision.datasets.CIFAR10} بارگذاری شده است. تصاویر آموزشی و آزمون دانلود شده و سپس مقادیر پیکسل آن‌ها به بازه [0, 1] نرمال‌سازی شده‌اند با تقسیم بر 255.0. این مرحله تضمین می‌کند که مقادیر ورودی در یک مقیاس استاندارد قرار دارند، که برای فرآیند آموزش مدل‌های یادگیری ماشین ضروری است. علاوه بر این، میانگین تصاویر آموزشی از هر دو مجموعه داده آموزشی و آزمون کم شده است تا داده‌ها به سمت صفر مرکزی شوند (mean centering).
	
	\textbf{2. بازبرچسب‌گذاری داده‌ها}
	
	برای سناریوی طبقه‌بندی دودویی، برچسب‌های اصلی CIFAR-10 تغییر داده شده‌اند. اگر یک تصویر متعلق به کلاس "هواپیما" (برچسب 0) باشد، برچسب آن به 1 تغییر می‌کند؛ در غیر این صورت، برچسب آن به 0 تنظیم می‌شود. این تبدیل، مجموعه داده را برای طبقه‌بندی دودویی آماده می‌کند، جایی که 1 نشان‌دهنده کلاس مثبت (هواپیما) و 0 نشان‌دهنده کلاس منفی (سایر) است.
	
	\textbf{3. مسطح کردن (Flatten) تصاویر}
	
	تصاویر CIFAR-10 که در ابتدا دارای ابعاد (ارتفاع, عرض, کانال) هستند، به بردارهای یک‌بعدی مسطح شده‌اند. برای تصاویر CIFAR-10 که ابعاد آن‌ها 32x32 پیکسل با 3 کانال رنگی (RGB) است، هر تصویر به یک بردار با 3072 ویژگی (32 * 32 * 3) تبدیل می‌شود. این فرمت برای ورودی به مدل رگرسیون لجستیک مناسب است.
	
	\textbf{4. پیاده‌سازی توابع فعال‌سازی و زیان}
	
	\begin{itemize}
		\item \textbf{تابع فعال‌سازی سیگموئید ($\sigma(z)$)}: این تابع احتمال خروجی یک نورون در مدل رگرسیون لجستیک را محاسبه می‌کند. خروجی آن بین 0 و 1 قرار دارد. فرمول پیاده‌سازی شده:
		$$\sigma(z) = \frac{1}{1 + e^{-z}}$$
		
		\item \textbf{تابع زیان کراس-انتروپی دودویی (Binary Cross-Entropy Loss)}: این تابع زیان برای مسائل طبقه‌بندی دودویی استفاده می‌شود و میزان تفاوت بین توزیع احتمال پیش‌بینی شده و توزیع واقعی را اندازه‌گیری می‌کند. یک مقدار اپسیلون (1e-10) برای جلوگیری از لگاریتم صفر اضافه شده است. فرمول پیاده‌سازی شده:
		$$L(y_{true}, y_{pred}) = -\frac{1}{N} \sum_{i=1}^{N} [y_{true,i} \log(y_{pred,i}) + (1 - y_{true,i}) \log(1 - y_{pred,i})]$$
		
	\end{itemize}
	
	\textbf{5. پیاده‌سازی آموزش مدل با استفاده از گرادیان کاهشی}
	
	کلاس \texttt{LogisticRegression} یک مدل رگرسیون لجستیک را پیاده‌سازی می‌کند.
	
	\begin{itemize}
		\item \textbf{مقداردهی اولیه پارامترها}: وزن‌ها ($W$) با مقادیر تصادفی کوچک (ضربدر 0.01) و بایاس ($b$) با صفر مقداردهی اولیه می‌شوند.
		\item \textbf{پیش‌بینی احتمالات (predict\_probabilities)}: این متد با استفاده از فرمول رگرسیون لجستیک، احتمالات خروجی را محاسبه می‌کند: $z = XW + b$ و $a = \sigma(z)$.
		\item \textbf{محاسبه گرادیان‌ها (compute\_gradients)}: گرادیان‌های مربوط به وزن‌ها ($dW$) و بایاس ($db$) با استفاده از فرمول‌های گرادیان برای کراس-انتروپی دودویی محاسبه می‌شوند:
		
		
		$dz = y_{pred} - y_{true}$
		
		
		$dW = \frac{1}{m} X^T dz$
		
		
		$db = \frac{1}{m} \sum dz$
		
		که در آن $m$ تعداد نمونه‌ها در بچ است.
		\item \textbf{به‌روزرسانی پارامترها (update\_params)}: وزن‌ها و بایاس با استفاده از نرخ یادگیری ($lr$) و گرادیان‌های محاسبه شده به‌روزرسانی می‌شوند:
		$W \leftarrow W - lr \cdot dW$
		
		
		$b \leftarrow b - lr \cdot db$
		
		\item \textbf{آموزش (train)}: متد \texttt{train} مدل را برای تعداد مشخصی از اپوک‌ها بر روی داده‌های آموزشی آموزش می‌دهد. این متد بر روی دسته‌های کوچک داده (mini-batches) تکرار می‌کند، احتمالات را پیش‌بینی می‌کند، زیان را محاسبه می‌کند، گرادیان‌ها را محاسبه کرده و پارامترها را به‌روزرسانی می‌کند. میانگین زیان و دقت هر اپوک را نیز ذخیره می‌کند.
		\item \textbf{پیش‌بینی (predict)}: این متد بر اساس آستانه 0.5، پیش‌بینی‌های دودویی (0 یا 1) را انجام می‌دهد.
	\end{itemize}
	
	برای دسته‌بندی داده‌ها به صورت دسته‌ای، یک کلاس \texttt{SimpleLoader} پیاده‌سازی شده است که قابلیت دسته‌بندی و ترکیب تصادفی (shuffling) داده‌ها را فراهم می‌کند.
	
	\textbf{6. ارزیابی مدل}
	
	مدل پس از آموزش با استفاده از معیارهای \texttt{confusion\_matrix} و \texttt{f1\_score} بر روی مجموعه داده آزمون ارزیابی شد. همچنین یک تابع \texttt{classification\_report} برای نمایش دقیق‌تر معیارهای \texttt{precision}, \texttt{recall} و \texttt{f1-score} برای هر کلاس استفاده شده است.
	
	برای یافتن بهترین آستانه طبقه‌بندی (threshold) که عملکرد مدل را به حداکثر می‌رساند، یک جستجو در بازه \lr{[0.1, 0.9]} با 17 نقطه (intervals) انجام شد. بهترین آستانه بر اساس بالاترین امتیاز F1-score انتخاب گردید.
	
	\textbf{نتایج ارزیابی:}
	
	پس از 200 اپوک با نرخ یادگیری \lr{0.0001} و اندازه بچ 512:
	
	\begin{itemize}
		\item \textbf{بهترین آستانه}: \lr{0.50}
		\item \textbf{بهترین F1-score}: \lr{0.4460}
		\item \textbf{دقت آزمون (\lr{Test Accuracy})}: \lr{0.8872}
	\end{itemize}
	
	\textbf{ماتریس درهم‌ریختگی (\lr{Confusion Matrix}):}
	
	\begin{tabular}{|l|l|l|}
		\hline
		& Pred Other & Pred Airplane \\ \hline
		True Other    & 8418       & 582           \\ \hline
		True Airplane & 546        & 454           \\ \hline
	\end{tabular}
	
	
	\textbf{گزارش طبقه‌بندی (\lr{Classification Report}):}
	
	\begin{tabular}{|l|l|l|l|l|}
		\hline
		& precision & recall & f1-score & support \\ \hline
		Other        & \lr{0.9391}    & \lr{0.9353} & \lr{0.9372 }  & \lr{9000}    \\ \hline
		Airplane     & \lr{0.4382}    & \lr{0.4540} &\lr{ 0.4460}   & \lr{1000}    \\ \hline
		\textbf{accuracy} &           &        & \textbf{\lr{0.8872}} & \lr{10000}   \\ \hline
		macro avg    & \lr{0.6887 }   & \lr{0.6947} & \lr{0.6916}   & \lr{10000}   \\ \hline
		weighted avg & \lr{0.8890}    & \lr{0.8872} &\lr{ 0.8881}   & \lr{10000}   \\ \hline
	\end{tabular}
	
	
	\textbf{تحلیل نتایج:}
	
	دقت کلی مدل \lr{ 0.8872} است که نشان‌دهنده عملکرد نسبتاً خوب مدل در طبقه‌بندی صحیح تصاویر است. با این حال، با بررسی \texttt{classification\_report}، مشخص می‌شود که عملکرد مدل برای کلاس "Other" (سایر) بسیار بهتر از کلاس "Airplane" (هواپیما) است.
	
	\begin{itemize}
		\item \textbf{کلاس "Other"}: دارای precision، recall و F1-score بالایی است (حدود \lr{0.93}). این نشان می‌دهد که مدل در شناسایی تصاویر "سایر" هم دقیق است (تعداد کمی از "هواپیما"ها را به اشتباه "سایر" دسته‌بندی می‌کند) و هم پوشش خوبی دارد (اکثر تصاویر "سایر" را به درستی شناسایی می‌کند).
		\item \textbf{کلاس "Airplane"}: دارای precision، recall و F1-score به مراتب پایین‌تری است (حدود \lr{0.44}). این نشان می‌دهد که مدل در شناسایی تصاویر "هواپیما" دقت و پوشش کمتری دارد. به عبارت دیگر، تعداد قابل توجهی از تصاویر "هواپیما" به اشتباه به عنوان "سایر" دسته‌بندی شده‌اند (recall پایین) و از بین تصاویری که مدل به عنوان "هواپیما" پیش‌بینی کرده، تعداد زیادی در واقع "سایر" بوده‌اند (precision پایین).
	\end{itemize}
	
	\textbf{چالش‌ها و بهبودهای احتمالی:}
	
	عملکرد ضعیف برای کلاس "هواپیما" می‌تواند ناشی از عدم توازن داده‌ها باشد؛ تعداد نمونه‌های کلاس "هواپیما" (1000) بسیار کمتر از کلاس "سایر" (9000) است.
	
	\textbf{نمودارهای زیان و دقت آموزش:}
	
	نمودارها نشان‌دهنده کاهش پیوسته زیان آموزش و افزایش دقت آموزش در طول اپوک‌ها هستند، که حاکی از یادگیری مدل است. نمودار ها در داخل کد دیده می شوند.
	
	
	\textbf{نتیجه:}
	
	مدل رگرسیون لجستیک به خوبی برای طبقه‌بندی دودویی تصاویر CIFAR-10 پیاده‌سازی و آموزش داده شد. با وجود دقت کلی مناسب، مدل در شناسایی کلاس اقلیت ("هواپیما") چالش‌هایی دارد. این موضوع می‌تواند در بخش‌های بعدی با استفاده از شبکه‌های عمیق‌تر و راهکارهای مدیریت عدم توازن داده‌ها بهبود یابد.
		
	\item بخش 2
	\item بخش3
	\item بخش 4
	\item بخش 5
\end{enumerate}