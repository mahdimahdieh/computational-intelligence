\section{ کدزنی و پیاده سازی‬}
\begin{enumerate}
	\item \textbf{رگرسیون لجستیک – آیا تصویر یک هواپیماست؟} \\
		در این بخش، یک مدل ساده رگرسیون لجستیک به‌صورت دستی پیاده‌سازی شد تا مشخص کند آیا یک تصویر از دیتاست CIFAR-10 مربوط به کلاس "هواپیما" (برچسب 0) است یا خیر (کلاس‌های دیگر). روند انجام کار به‌صورت زیر بود:
		
	\begin{itemize}
		\item \textbf{بارگذاری و نرمال‌سازی داده‌ها:} داده‌های آموزشی و تست از دیتاست CIFAR-10 بارگذاری و مقدار پیکسل‌های آن به بازه $[0,1]$ نرمال‌سازی شد.
			
		\item \textbf{بازبرچسب‌گذاری:} برچسب‌ها به‌صورت دودویی بازنویسی شدند؛ تصاویر کلاس هواپیما مقدار 1 گرفتند و سایر کلاس‌ها مقدار 0.
			
		\item \textbf{تبدیل شکل تصاویر:} هر تصویر سه‌بعدی با اندازه $32 \times 32 \times 3$ به یک بردار 3072‌بعدی (Flatten) تبدیل شد تا برای مدل خطی قابل استفاده باشد.
			
		\item \textbf{تابع فعال‌سازی و تابع هزینه:} از تابع سیگموید برای فعال‌سازی و از تابع کراس‌انتروپی دودویی به‌عنوان تابع هزینه استفاده شد:
			\[
			\hat{y} = \frac{1}{1 + e^{-z}}, \quad
			\mathcal{L}(\hat{y}, y) = -[y \log \hat{y} + (1-y) \log (1-\hat{y})]
			\]
			
		\item \textbf{آموزش مدل با گرادیان کاهشی:} وزن‌ها و بایاس به‌صورت تصادفی مقداردهی اولیه شدند و طی چندین ایپاک با استفاده از الگوریتم Gradient Descent به‌روزرسانی شدند. نرخ یادگیری برابر 0.01 در نظر گرفته شد.
			
		\item \textbf{ارزیابی مدل:} عملکرد مدل روی داده‌های تست با استفاده از ماتریس سردرگمی و معیار F1-score بررسی شد. نتایج به‌صورت زیر بود:
		\begin{itemize}
			\item دقت (Precision): 0.727
			\item یادآوری (Recall): 0.533
			\item امتیاز F1: تقریباً 0.615
		\end{itemize}
			
		\item \textbf{چالش‌ها و تحلیل:}
		\begin{itemize}
			\item سادگی بیش از حد مدل باعث عملکرد ضعیف در تشخیص ویژگی‌های پیچیده تصاویر رنگی شد.
			\item مدل تنها قادر به یافتن مرز تصمیم خطی بین دو کلاس بود و برای تصاویر با شباهت بالا به هواپیما اشتباه تشخیص می‌داد.
			\item پیشنهاد می‌شود برای بهبود عملکرد، از لایه‌های پنهان و معماری‌های پیچیده‌تر استفاده شود (بخش‌های بعدی تمرین به این موضوع می‌پردازند).
		\end{itemize}
	\end{itemize}
		
	\item بخش 2
	\item بخش3
	\item بخش 4
	\item بخش 5
\end{enumerate}