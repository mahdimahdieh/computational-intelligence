% Exercise Template
% بخش پنج م: پیاده‌سازی CNN با استفاده از PyTorch
% By: Generated by ChatGPT

\documentclass[12pt]{exam}

\usepackage{setspace}
\usepackage{listings}
\usepackage{xcolor}
\usepackage{xepersian}

% Define colors
\definecolor{codegreen}{rgb}{0,0.6,0}
\definecolor{codegray}{rgb}{0.5,0.5,0.5}
\definecolor{codepurple}{rgb}{0.58,0,0.82}
\definecolor{backcolour}{rgb}{0.95,0.95,0.92}

% Configure listings style
\lstdefinestyle{mystyle}{
	backgroundcolor=\color{backcolour},   
	commentstyle=\color{codegreen},
	keywordstyle=\color{magenta},
	numberstyle=\tiny\color{codegray},
	stringstyle=\color{codepurple},
	basicstyle=\ttfamily\footnotesize\setLTR,
	breakatwhitespace=true,
	breaklines=true,
	captionpos=top,
	keepspaces=true,
	numbers=left,
	numbersep=5pt,
	showspaces=false,
	showstringspaces=false,
	showtabs=false,
	tabsize=2,
	frame=single,
	abovecaptionskip=5pt,
	belowcaptionskip=5pt,
}

\lstset{style=mystyle}
\renewcommand{\lstlistingname}{برنامه}
\usepackage{graphicx,subfigure,wrapfig}
\usepackage{float}
\usepackage{multirow}
\usepackage{pgf-pie}
\usepackage{etoolbox}
\usepackage[margin=20mm]{geometry}
\usepackage{hyperref}
\hypersetup{
	colorlinks=true,
	linkcolor=blue,
	filecolor=magenta,
	urlcolor=cyan,
	pdftitle={بخش پنج م: پیاده‌سازی CNN با PyTorch},
	pdfpagemode=FullScreen,
}
\settextfont{XB Niloofar}

\begin{document}
	
	\vspace{1em}
	
	در این بخش شما باید یک شبکه عصبی کانولوشنال (CNN) را برای دسته‌بندی تصاویر CIFAR-10 با استفاده از PyTorch پیاده‌سازی کنید و عملکرد آن را با پرسپترون چندلایه قبلی مقایسه نمایید.
	
	\begin{questions}
		
		% Task 1
		\question
		\textbf{تنظیم داده‌ها و بارگذاری CIFAR-10}
		
		مجموعه‌داده CIFAR-10 را با نرمال‌سازی مناسب بارگذاری کنید و DataLoader‌‌های آموزش و تست را با اندازه بچ و شافل دلخواه ایجاد نمایید.
		
		\lstinputlisting[language=Python, caption=تعریف \lr{transforms} و \lr{DataLoader}]{./scripts/data_loader_cifar10.py}
		
		
		% Task 2
		\question
		\textbf{تعریف معماری ساده CNN}
		
		یک کلاس PyTorch از نوع \lr{nn.Module} با ساختار زیر پیاده‌سازی کنید:
		\begin{itemize}
			\item \lr{Conv2d} با 32 فیلتر، کرنل $3\times3$, فعال‌سازی \lr{ReLU}
			\item \lr{MaxPool2d} با کرنل $2\times2$
			\item \lr{Conv2d} با 64 فیلتر، کرنل $3\times3$, فعال‌سازی \lr{ReLU}
			\item \lr{MaxPool2d} با کرنل $2\times2$
			\item \lr{Flatten()} و دو لایه تمام‌متصل 128 و 10 نورونی (Softmax خروجی)
		\end{itemize}
		
		\lstinputlisting[language=Python, caption=کلاس \lr{SimpleCNN}]{./scripts/simple_cnn.py}
		
		
		% Task 3
		\question
		\textbf{حلقه آموزش مدل}
		
		حلقه آموزشی را با تابع هزینه \lr{CrossEntropyLoss} و بهینه‌ساز \lr{SGD} با مومنتم پیاده کنید. در هر صد گام آموزشی، میانگین لوس را چاپ نمایید.
		
		\lstinputlisting[language=Python, caption=حلقه آموزش \lr{train loop}]{./scripts/train_cnn.py}
		
		
		% Task 4
		\question
		\textbf{ارزیابی مدل و گزارش دقت}
		
		مدل آموزش‌دیده را روی مجموعه تست ارزیابی کنید و دقت نهایی را محاسبه و چاپ کنید. سپس وزن‌های مدل را ذخیره نمایید.
		
		\lstinputlisting[language=Python, caption=ارزیابی و ذخیره مدل]{./scripts/evaluate_cnn.py}
		
		
		% Task 5
		\question
		\textbf{گزارش معماری و تحلیل نتایج}
		
		یک گزارش مختصر بنویسید که شامل معماری نهایی، جزییات پیاده‌سازی، روند آموزش و نتایج ارزیابی باشد. همچنین مزایا و معایب استفاده از CNN را در مقایسه با پرسپترون چندلایه مورد بحث قرار دهید.
		
		\question
		\textbf{مزایا و معایب استفاده از \lr{CNN} در مقابل یک پرسپترون چندلایه}
		
		\vspace{0.5em}
		\noindent
		\textbf{مزایا:}
		\begin{itemize}
			\item \textbf{استخراج ویژگی‌های مکانی:} لایه‌های کانولوشنال با حفظ ساختار دوبعدی تصویر، الگوهای محلی مانند لبه‌ها و بافت‌ها را بهتر استخراج می‌کنند، در حالی که MLP تصویر را صاف‌شده دریافت می‌کند.
			\item \textbf{کاهش پارامترها:} با اشتراک وزن در فیلترها، تعداد پارامترها به‌طور قابل‌توجهی کمتر از MLP بوده و خطر بیش‌برازش کاهش می‌یابد.
			\item \textbf{مقیاس‌پذیری:} استفاده از pooling و فیلترهای محلی باعث می‌شود CNN بتواند روی تصاویر بزرگ‌تر یا پیچیده‌تر نیز کارایی مناسبی داشته باشد.
			\item \textbf{تعمیم بهتر:} CNN در برابر تغییرات کوچک در تصویر (جابجایی، چرخش، تغییر نور) مقاوم‌تر است و به همین دلیل در داده‌های بصری عملکرد بهتری دارد.
		\end{itemize}
		
		\noindent
		\textbf{معایب:}
		\begin{itemize}
			\item \textbf{پیچیدگی پیاده‌سازی:} طراحی معماری CNN اعم از انتخاب تعداد لایه‌ها، فیلترها و سایر ابرپارامترها نیازمند تجربه و آزمون‌وخطای بیشتری است.
			\item \textbf{زمان آموزش:} محاسبات کانولوشن هزینه‌برتر از ضرب‌های برداری ساده در MLP بوده و ممکن است زمان آموزش طولانی‌تری داشته باشد.
			\item \textbf{وابستگی به GPU:} برای آموزش سریع و مؤثر CNN معمولاً نیاز به شتاب‌دهنده‌هایی مانند GPU است، در حالی که MLP را می‌توان حتی روی CPU اجرا کرد.
		\end{itemize}
		
	\end{questions}
	
\end{document}
